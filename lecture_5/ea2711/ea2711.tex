
\section{Guest Lecture}

\subsection{What is data visualization?}
\begin{itemize}
  \item "The use of computer-generated interactive visual representations to amplify cognition."
  \item The goal of visualizations is to develop and support hypothesis, and to inspire and convince others
\end{itemize}

\subsection{Exploratory data analysis}
There are multiple methods of showing data
\begin{itemize}
  \item Plain data -- very difficult to perceive patterns and make sense of data
  \item Summary-statistics, e.g. mean, median, etc. -- can get an intuitive understanding
  \item Plot Data -- can identify patterns and trends
\end{itemize}
To create a visualization, we must record data, analyze data, and communicate the findings to the community.

\subsection{What makes "good" visualizations?}

Identify which dimensions of data should be matched to which areas
\subsubsection{Design Principles}
\begin{itemize}
    \item Expressiveness
        \begin{itemize}
            \item express all the facts/info in the set of data, nothing less and nothing more
            \item choose correct type of chart -- ex. a scatter plot cannot express a one to many relation
        \end{itemize} 
    \item Effectiveness
    \begin{itemize}
        \item More effective if information can be conveyed more quickly
        \item Easy to understand
    \end{itemize} 
\end{itemize}
Some things to remember about visualizations:
\begin{itemize}
    \item Tell the truth and nothing but the truth -- i.e. Don't lie!
    \item Use encodings that people decode better
    \item Not all visual encodings are equal
    \begin{itemize}
        \item Some may contain biases for example
    \end{itemize}
\end{itemize}

\subsection{Visualization Effectiveness}
Steven's Power law \[S = I^p\]
where S is the perceived sensation, I is the physical intensity and p is the exponential relation. 
\subsubsection{Perception Biases}
\begin{itemize}
    \item Area - we underestimate large areas over small areas 
    \item Perception of shock increases quicker than the actual level of shock
    \item We perceive length and position well, much better than color saturation or pie charts (Cleveland and McGill Experiment)
    \item We can actually rank human perception biases
\end{itemize}

\subsubsection{Data Types}
\begin{itemize}
    \item \textbf{Normal} -- non intrinsic ordering -- eye color, gender, etc.
    \item \textbf{Ordinal} -- contains a natural ordering -- socioeconomic class, month, etc.
    \item \textbf{Quantitative} -- is described numerically
\end{itemize}

\subsubsection{Other Decisions}
\underline{Color}
\begin{itemize}
    \item How should I color my plot? -- not all colors are equal
    \item Choose colors that maintain distinguishability
    \item Small changes in color should correlate with proportional changes in value
\end{itemize}
\underline{Tools}
\begin{itemize}
    \item There is a tradeoff between speed and expressiveness
    \item Declarative Encoding Languages
    \begin{itemize}
        \item program by describing \textit{what} not \textit{how}
        \item separate specification from execution
        \item examples are: HTML/CSS, SQL, D3
        \item Advantages
        \begin{itemize}
            \item faster iteration
            \item performance
            \item reuse-ability 
            \item portability
        \end{itemize}
        \item Disadvantages
        \begin{itemize}
            \item debugging is difficult
        \end{itemize}
    \end{itemize}
    \item The Grammar of Graphics
    \begin{itemize}
        \item Set of principles for graphical APIs
        \item "Don't give a pie, give primitives to make a pie and more"
        \item Provide small tools that provide more flexibility and customization
    \end{itemize}
\end{itemize}

\section{ggplot}
Visit the \href{https://github.com/jhofman/msd2017/blob/master/lectures/lecture_5/visualization_with_ggplot2.ipynb}{\underline{Jupyter Notebook}} for exact source code, some observations are listed here.
\begin{itemize}
    \item Purpose of a plot is to communicate a ~10 word point 
    \item Use geoms to represent data points, use aes() function to add aesthetics, variables, axes, etc. 
    \item Pipe commands using '+' not '\%$>$\%' -- the data frame will default to first argument
    \item When using aes(), order of arguments doesn't matter -- they are added just as descriptors
    \item \textbf{geom\_histogram()} -- implicit stat counting happening, then maps the count for you
    \begin{itemize}
        \item Be sure to specify the number of bins -- If you don't, some random number will be assumed and a warning will be thrown    
        \item Identifies categorical variabels and maps them to bins if needed
    \end{itemize}
    \item \textbf{geom\_smooth()} -- fits a model to the data
    \begin{itemize}
        \item specify method="ln" to force linear model
    \end{itemize}
    \item \textbf{geom\_density()} -- fill in plot
    \item Be careful what to include in aes() -- if it is a constant, best to keep it out of the aesthetic mappings
    \item Be careful when using xlim -- may either zoom into the plot, or eliminate points from computation
    \item R will default categorical axes to alphabetical ordering -- often it is better to change this to something that conveys pattern/point better
\end{itemize}
