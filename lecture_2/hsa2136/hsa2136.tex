%----------------------------------------
% Write your notes here
%----------------------------------------

\section{Counting}


Regular Statistics is basically just conditional probability: trying to find $P(y | x)$.  If we have a small sample size, we're going to have large expected error.  

The uncertainty we get is inversely related to the square root of the sample size:
\[
\sqrt{ \dfrac{p(1-p)}{N} }
\]

The problem gets worse when we condition on more variables.  For example, if we condition on multiple features such as $P(y | x_1, x_2, ...)$

Concrete example: 100 ages, 2 sexes, 5 races, 3 parties $\rightarrow$ 3000 groups.  So, if we want to have a good enough $N$ for each group, we need LOTS of samples.

One thing to do: bin features e.g. age between 18-24 is a group.  

Another solution is to make a huge, non-representative study - e.g. poll all XBox users (mostly 9-17 year old guys).  This causes a new problem: computation is very hard on such a large sample.  

New Framework: split/apply/combine: split the data into groups, apply the computation (e.g. mean), combine the groups to get a sample-wide statistic.  

Examples: 

Bad Way: 
\begin{enumerate}
  \item Scan through X searching for $a$
  \item Compile list of y-values
  \item compute mean
  \item repeat for b, c, ... etc.
\end{enumerate}

Time: N*G (\# of samples times \# of groups)

Space: N\\

Better Way: 
\begin{enumerate}
  \item Scan through X 
  \item Compile list of y-values for each value of x
  \item compute mean
\end{enumerate}

Time: 2N

Space: 2N\\

Even Better:
\begin{enumerate}
  \item Scan through X 
  \item Sum up total for each group and a counter
  \item compute mean
\end{enumerate}


Time: 2N

Space: 2G\\